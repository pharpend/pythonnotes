\chapter{Introduction}

Hello, these are lecture notes for Python. The first ``lecture'' is
just getting set up. The second lecture is toying around with the
Read/Eval/Print Loop (REPL). The third lecture writing our first
actual Python program.

Throughout this text, many references will be made to the CodingBat
problems: \url{http://codingbat.com/python}. It's very important that
you manually type in and personally test every piece of code that you
see here. Do not skim the book, and do not copy \& paste the code
examples. Manually type them out. It really helps with understanding
the material.

\subsection{Conventions used in the book}

If you see \code{monospaced text}, it refers to a URL or code. If you
are reading this as a PDF, you can click on URLs, footnotes,
bibliography citations, etc, and your PDF reader will take you to the
appropriate place. Long blocks of monospace text will have their own
fancy box with line numbering:

\begin{lstlisting}[language=Python]
"Hi, I'm a string inside a code block."
\end{lstlisting}

Many of the code blocks have syntax highlighting. The highlighting
doesn't really mean anything, it's just there to help with
readability. 

If you see a bunch of code preceded by a \code{%} or a \code{$}, then
  you're supposed to run the stuff afterward in a terminal. There may
  be following lines to indicate the expected result.

\begin{lstlisting}
% uname -a
Linux tsarina 3.16.0-4-amd64 #1 SMP Debian 3.16.7-ckt25-2 (2016-04-08) x86_64 GNU/Linux
\end{lstlisting}

If you see a bunch of monospace text preceded by a 

\section{Setup}

\subsection{Installing Python}

In this section, we're going to get Python installed on your system,
and run the famous \code{hello, world} program. The setup varies by
operating system.

\paragraph{Windows}

Navigate yourself over to \url{https://www.python.org/downloads/}, and
download \& install the latest version of Python 3.

\paragraph{Unix}

If you are on Linux, BSD, or a Macintosh, you likely already have
Python installed. We'll be using Python 3. Open up a terminal, and run
\code{python --version}. On my system (Debian 8.5):

\begin{lstlisting}
% python --version
Python 2.7.9
\end{lstlisting}

Oh, by the way, a line preceded with a \code{%} sign indicates that
the line should be entered into a terminal. Anyway, proceeding. On my
system, Python 2 is the default. However, the command \code{python3}
exists, and upon running \code{python3 --version}, I get

\begin{lstlisting}
% python3 --version
Python 3.4.2
\end{lstlisting}

If not, consult your operating system's documentation for installing
Python.

\subsection{Text editor}

You'll also want a text editor.

\begin{enumerate}
\item For the purposes of this text, Kate
  (\url{https://kate-editor.org/}) is my recommendation, because it's
  fancy, easy-to-use, and it's easily installable on every operating
  system.
\item I personally use GNU Emacs
  (\url{https://www.gnu.org/software/emacs/}). However, it's quite
  difficult to set up, and doing so is well outside the scope of this
  text.
\item Vim (\url{http://www.vim.org/}) is a very popular editor, and is
  very nice. It's a bit easier to use and set up than Emacs. However,
  it's still miles behind Kate.
\end{enumerate}

\subsection{Hello world}

The \code{hello, world} program is a program that's featured in pretty
much every programming text. The program literally prints out
\code{hello, world} in the terminal console when you run
it.\footnote{I believe the program first appeared in the classic book
  \emph{The C Programming Language}, by Kernighan \& Richie. A full
  citation can be found in the bibliography. \cite{k-and-r}} We'll
write an actual program later. However, for now, we'll just have the
REPL do something similar to what the program will hypothetically do.

\begin{lstlisting}
>>> print('hello, world')
hello, world
\end{lstlisting}

The \code{>>>} indicates that you should manually type in whatever
follows the \code{>>>}. Boom, we're done!

\section{Toying around in the REPL}

\term{REPL} stands for Read/Evaluate/Print Loop. In a little while,
we'll learn what all of those words mean. For now, just know that it's
the interactive console thing that 
