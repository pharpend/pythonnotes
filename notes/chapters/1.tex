\chapter{Introduction}

Hello, these are lecture notes for Python. The first ``lecture'' is
just getting set up. The second lecture is toying around with the
Read/Eval/Print Loop (REPL). The third lecture writing our first
actual Python program.

Throughout these notes, many references will be made to the CodingBat
problems: \url{http://codingbat.com/python}.

\section{Setup}

In this section, we're going to get Python installed on your system,
and run the famous \code{hello, world} program. The setup varies by
operating system.

\paragraph{Windows}

Navigate yourself over to \url{https://www.python.org/downloads/}, and
download \& install the latest version of Python 3.

Then, open up your start menu, and find a program called IDLE. Open
that. You should see a bunch of garbage telling you the Python
version, what operating system, etc. Then, you'll see three ``greater
than'' signs, followed by a space. I'm on Linux, so my thing will be a
bit different. However, here's what it looks like on my system.

\begin{lstlisting}
Python 3.4.2 (default, Oct  8 2014, 10:45:20) 
[GCC 4.9.1] on linux
Type "help", "copyright", "credits" or "license" for more information.
>>> 
\end{lstlisting}

The \code{hello, world} program is a program that's featured in pretty
much every programming text. The program literally prints out
\code{hello, world} in the terminal console when you run
it.\footnote{I believe the program first appeared in the classic book
  \fullcite{k-and-r}.} We'll write an actual program later. However,
for now, we'll just have the REPL do something similar.

If you are on Linux, BSD, or a Macintosh, you likely already have
Python installed. We'll be using Python 3. Open up a terminal, and run
\code{python --version}. On my system (Debian 8.5):

\begin{lstlisting}
% python --version
Python 2.7.9
\end{lstlisting}

Oh, by the way, a line preceded with a \code{%} sign indicates that
the line should be entered into a terminal. Anyway, proceeding. On my
system, Python 2 is the default. However, the command \code{python3}
exists, and upon running \code{python3 --version}, I get

\begin{lstlisting}
% python3 --version
Python 3.4.2
\end{lstlisting}

If not, consult your operating system's documentation for installing
Python.

\subsection{Hello World}

Now that we have Python installed, open up a terminal, and run the
appropriate command (either \code{python} or \code{python3}), and type
in the following line:




\section{Toying around in the REPL}

\term{REPL} stands for Read/Evaluate/Print Loop. In a little while,
we'll learn what all of those words mean. For now, just know that it's
the interactive console thing that 
