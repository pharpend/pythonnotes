\chapter{Introduction}

Hello, these are lecture notes for Python. The first ``lecture'' is
just getting set up. The second lecture is toying around with the
Read/Eval/Print Loop (REPL). The third lecture writing our first
actual Python program.

Throughout this text, many references will be made to the CodingBat
problems: \url{http://codingbat.com/python}. It's very important that
you manually type in and personally test every piece of code that you
see here. \textbf{Do not skim the book, and do not copy \& paste the
  code examples. Manually type them out.} It really helps with
understanding the material.

\subsection{Conventions used in the book}

If you see \code{monospaced text}, it refers to a URL or code. If you
are reading this as a PDF, you can click on URLs, footnotes,
bibliography citations, etc, and your PDF reader will take you to the
appropriate place. Long blocks of monospace text will have their own
fancy box with line numbering:

\begin{lstlisting}[language=Python]
"Hi, I'm a string inside a code block."
\end{lstlisting}

Many of the code blocks have syntax highlighting. The highlighting
doesn't really mean anything, it's just there to help with
readability. 

If you see a bunch of stuff preceded by a
dollar sign, then you're supposed to run the stuff afterward in a
terminal. There may be following lines to indicate the expected
result. Do not type in the dollar sign.

\lstinputlisting{listings/uname-a.txt}

If you see a bunch of monospace text preceded by a \code{>>>}, then
you are supposed to run that in the interactive Python console. I'll
explain what that is in a second.

\subsection{Mathematical notation}

I'm a mathematician, so I tend to think of---and state
things---mathematically. Therefore, I'm going to introduce some basic
mathematical notation so that you will have some idea of what I'm
talking about.

\begin{definition}
  A \term{set} is an collection of objects, which are called
  \term{elements}. \cite{taylor-fnds}
\end{definition}

Given a set $A$, to say \ctext{$x$ is an element of $A$,} we write
$$x \in A.$$ Alternatively, you can just read $x \in A$ as ``$x$ in
$A$'' If \ctext{$x$ is not an element of $A$,} we write
$$x \notin A.$$ Sometimes, I'll use the shorthand $$x, y \in A,$$
because I'm too lazy to type out \ctext{$x \in A$ and $y \in A$.}

For sufficiently small sets, the typical way to define them is to list
their elements between curly braces. For instance,
\ctext{$\mset{1,2,3}$ is a set.} Order and duplication do not matter, so
$$\mset{1,2,3,1,1,2,3,2,3,1,1}$$ and $$\mset{2,1,3}$$ are both the same
set. We then have $$1 \in \mset{1,2,3},$$ but
$$4 \notin \mset{1,2,3}.$$

\begin{definition}
  The number of elements in a set is called the \term{cardinality} of
  the set. Given a set $A$, the cardinality of $A$ is denoted
  $\abs{A}$.
\end{definition}

For instance, \ctext{if $A = \mset{1,2,3},$ then $\abs{A} = 3.$} For
\ctext{$B = \mset{1,2,3,4,5,6},$ $\abs{B} = 6$.}

\begin{definition}
  A set $A$ is a \term{subset} of $B$, denoted $$A \subof B,$$ if
  every element of $A$ is also an element of $B$. That is, \ctext{for
    all $x \in A$, $x \in B$.}
\end{definition}

For instance, $$\mset{1,2,4} \subof \mset{1,2,3,4,5}.$$

\begin{definition}
  Two sets $A$ and $B$ are equal, written $A = B$, if
  \ctext{$A \subof B$, and $B \subof A$.}
\end{definition}

\begin{remark}
  The definition I gave is for \term{improper subsets}. That is,
  $A \subof B$ allows the possibility that $A = B$. The term
  \term{proper subset} refers to a subset that is strictly
  smaller. That is, \ctext{$A$ is a proper subset of $B$,} written
  $$A \psubof B,$$ if $A \subof B$, and there is some element
  $x \in B$ such that $x \notin A$.

  The notion of improper subsets turns out to be a lot more useful,
  and comes up a lot more. Therefore, when I say ``is a subset of'', I
  am referring to the notion of improper subsets.

  Some people use the notation $A \subset B$ to refer to proper
  subsets, and some use it to refer to improper subsets. It's
  therefore ambiguous notation, so I'm going to avoid using it.
\end{remark}

\begin{definition}
  The \term{difference} of two sets $A$ and $B$ is defined as
  $$A \setminus B = \scomp{x}{x \in A \text{, but } x \notin B}.$$
\end{definition}

If you aren't familiar with that notation: the $:$ should be read as
``such that''. The whole thing is therefore \ctext{the set of all $x$
  such that $x$ is in $A$, but $x$ is not in $B$.} It's called a
\term{set comprehension}. Later on, we'll explore \term{list
  comprehensions} in Python, which are similar.

Georg Cantor was the mathematician largely responsible for modern set
theory. One of his main results was in the development of
\term{infinite sets}, particularly in properties of the cardinalities
thereof. There are a lot of weird and downright counterintuitive
properties of the cardinalities of infinite sets. Said properties are
largely outside the scope of this book, but I'll mention a few as we
go along. For now, here are a few sets which we will use:

\begin{enumerate}
\item The \term{natural numbers}, denoted $\N$, are $$\N = \mset{1, 2,
  3, 4, \dots}.$$ Some people use $0$ as the first natural number. It
doesn't matter a whole lot.
\item The \term{integers}, denoted $\Z$ (for the German ``zahlen''),
  are $$\Z = \mset{0, \pm 1, \pm 2, \pm 3, \pm 4, \dots}.$$
\item The \term{quotient numbers}, also called (\term{rational
    numbers}\footnote{The term ``rational'' means ``of or relating to
    a ratio''. Although, the rational numbers do behave much more
    reasonably than their complement, the \term{irrational
      numbers}. (I know, I know\dots)}, or \term{fractions}) denoted
  $\Q$, are the set of all fractions of two integers (where the
  denominator is nonzero):
  $$\Q = \scomp{\frac{x}{y}}{x \in \Z, y \in \Z \setminus \mset{0}}.$$
\item The \term{real numbers}, denoted $\R$, are any conceivable number. For
  instance, the square root of $2$ is not a rational number, but is a
  real number. We'll prove this later on.
\item There are some weird holes in the real numbers, too. For
  instance, the square root of $-1$ is not a real number. Therefore,
  we have the complex numbers $\C$. Let $i = \sqrt{-1}$. Then, $$\C =
  \scomp{a + bi}{a, b \in \R}.$$
\end{enumerate}

I won't dive too far into rigorously defining these numbers. Mostly
because it's surprisingly difficult and not very interesting. If you
would like to learn more about them, I invite you to read
\cite{landau-fnds}, and perhaps \cite{flanigan-complex-vars}, if
you're interested in the complex numbers.

\section{Setup}

Okay, getting back to reality:

\subsection{Installing Python}

In this section, we're going to get Python installed on your system,
and run the famous \code{hello, world} program. The setup varies by
operating system.

\paragraph{Windows}

Navigate yourself over to \url{https://www.python.org/downloads/}, and
download \& install the latest version of Python 3. 

\paragraph{Unix}

If you are on Linux, BSD, or a Macintosh, you likely already have
Python installed. We'll be using Python 3. Open up a terminal, and run
\code{python --version}. On my system (Fedora 24):

\lstinputlisting{listings/python2version.txt}

Anyway, proceeding. On my system, Python 2 is the default. However, on
your system, Python 3 might be the default.

\lstinputlisting{listings/python3version.txt}

If, when you run \code{python --version}, the version starts with
\code{3}, then you can just run the \code{python} command when I tell
you to ``use the appropriate Python command,'' or something. If the
default version of Python is 2.something, then run \code{python3} to
get at Python.

If Python is not installed, consult your operating system's
documentation for installing Python. It's installed by default on
every single operating system I've used, except Windows.

\paragraph{For everyone}

When I say ``open the Python REPL'', I now mean ``open a terminal, and
run the appropriate Python command'' (whichever one gives you a
version of Python 3). Note, if you just run the plain \code{python}
(or \code{python3}) command, you get a nice \term{Read/Eval/Print
  Loop} (REPL):

\lstinputlisting{listings/python3repl.txt}

\subsection{Text editor}

You'll also want a text editor.

\begin{enumerate}
\item For the purposes of this text, Kate
  (\url{https://kate-editor.org/}) is my recommendation, because it's
  fancy, easy-to-use, and it's easily installable on every operating
  system.
\item I personally use GNU Emacs
  (\url{https://www.gnu.org/software/emacs/}). However, it's quite
  difficult to set up, and doing so is well outside the scope of this
  text.
\item Vim (\url{http://www.vim.org/}) is a very popular editor, and is
  very nice. It's a bit easier to use and set up than Emacs. However,
  it's still miles behind Kate, as far as ease-of-use is concerned.
\end{enumerate}

\subsection{Hello world}

The \code{hello, world} program is a program that's featured in pretty
much every programming text. The program literally prints out
\code{hello, world} in the terminal console when you run
it.\footnote{I believe the program first appeared in the classic book
  \fullcite{k-and-r}, commonly referred to as ``K\&R.''} We'll write
an actual program later. However, for now, we'll just have the REPL do
something similar to what the program will hypothetically do.

\begin{lstlisting}
>>> print('hello, world')
hello, world
\end{lstlisting}

Boom, we're done!

\section{Toying around in the REPL}

\term{REPL} stands for Read/Evaluate/Print Loop. In a little while,
we'll learn what all of those words mean. For now, just know that it's
the interactive console thing that we used to write a program called
``hello world''.

\paragraph{Python as a calculator}

The four elementary operations operate as you would expect in Python
3.

\begin{lstlisting}
>>> 2 + 2
4
>>> 23 + 2 + 13
38
>>> 428 / 4
107.0
>>> 84/14
6.0
\end{lstlisting}

\paragraph{Integer division} Python 2 uses ``integer division'', which
is, instead of allowing something like $\frac{2}{3}$ to be a quotient,
it insists that the result is an integer. Therefore, when you compute
\code{x / y} in Python 2, instead of $\frac{x}{y}$ as you would
normally think of it, Python 2 instead computes the floor of the
fraction.   Examples:

\begin{definition}
  Given two numbers $x, y \in \Z$, with $y \ne 0$, we say \ctext{$y$
    divides $x$} if there is some $c \in \Z$ such that $cy = x.$
\end{definition}

\begin{remark}
  All of the following statements are equivalent:

  \begin{enumerate}
  \item $x$ divides $y$,
  \item $y$ is divisible by $x$,
  \item $x$ is a factor of $y$,
  \item $\frac{x}{y}$ is an integer.
  \end{enumerate}
\end{remark}

\begin{definition}
  The \term{floor} of a number $x \in \R$, denoted $\floor{x}$ is
  defined as the greatest integer $y$ such that $y \le x$.
\end{definition}

Python 2 uses \term{floor division}, or \term{integer division}. Examples:

\begin{lstlisting}
% python2
Python 2.7.12 (default, Jul 18 2016, 10:55:51) 
[GCC 6.1.1 20160621 (Red Hat 6.1.1-3)] on linux2
Type "help", "copyright", "credits" or "license" for more information.
>>> 2/3
0
>>> -2/3
-1
>>> 4/3
1
>>> -4/3
-2
\end{lstlisting}

If $y \divides x$, then $\frac{x}{y}$ is an integer, so therefore
floor division works like normal division:

\begin{lstlisting}
>>> 10 / 5
2
>>> 24 / 6
4
\end{lstlisting}

We'll be using Python 3, which has sane division. However, it's worth
noting that Python 2 (and most programming languages) use floor
division. It's also a nice segue into introducing a feature in
Python. You can import features from future versions of Python. For
instance, to import sane division into Python 2, you can run

\begin{lstlisting}
>>> from __future__ import division
>>> 2 / 3
0.6666666666666666
\end{lstlisting}

\begin{remark}
  Note that there are a finite number of digits at the end. A
  recurring problem in computer science is \term{floating-point
    approximations}. Where, you have some ``pure'' mathematical value
  that you have to represent numerically. With some of them, like
  $\frac{1}{4} = 0.25,$ the numerical representation terminates. With
  others, like $\frac{1}{3} = 0.333\dots,$ the numerical
  representation does not terminate, but instead follows an obvious
  pattern.

  With irrational numbers (real numbers that are not quotients), the
  numerical representation does not follow any pattern, and never
  terminates. Examples of irrational numbers include
  Euler's\footnote{Euler is pronounced ``oiler.''} constant
  $e \approx 2.7182818$, and $\pi \approx 3.141592654$, which is the
  ratio of the circumference of a circle to its diameter. 

  It's even difficult to represent terminating rational numbers
  accurately. For instance:

\begin{lstlisting}
>>> 0.1 + 0.2
0.30000000000000004
\end{lstlisting}

  There are many cases where you need to be smarter than the
  computer. This is one of them.
\end{remark}

You should note, that in Python 3, if you compute \code{x / y}, even
if $y \divides x$, the result will be a ``floating-point'' number,
meaning it has a decimal point.

\begin{lstlisting}
>>> 42 / 7
6.0
>>> 45 / 9
5.0
\end{lstlisting}

The difference between an integer and a floating-point number is what
is known as a \term{type}. Types in programming are \emph{almost} the
same thing as sets in math. To find the type of something, you can
call the function \code{type()}:

\begin{lstlisting}
>>> type(5)
<class 'int'>
>>> type(5.0)
<class 'float>
>>> type(1 / 3)
<class 'float'>
>>> type(3 / 1)
<class 'float'>
\end{lstlisting}

Note that, even in Python 2, to compute $\frac{2}{3}$ using floating
division, you can just compute \code{2.0 / 3}, the result of which is
a \code{float}.

\subsection{Strings}

\term{String} is the programming term for text. In Python, strings are
denoted by either single or double quotes. It doesn't matter a whole
lot. I like single quotes better, so I'll try to use them.

\begin{lstlisting}
>>> 'The quick brown fox jumped over the lazy dog'
'The quick brown fox jumped over the lazy dog'
>>> "The quick brown fox jumped over the lazy dog"
'The quick brown fox jumped over the lazy dog'
\end{lstlisting}

The difference becomes apparent if your string contains either single
quotes or double quotes inside the string. For instance:

\begin{lstlisting}
>>> "Don't tread on me"
"Don't tread on me"
>>> 'Don't tread on me'
  File "<stdin>", line 1
    'Don't tread on me'
         ^
SyntaxError: invalid syntax
\end{lstlisting}

This is because Python interprets the apostrophe in \code{Don't} as
the end of the string, and then gets confused when there's a bunch of
garbage afterward. You can still insert an apostrophe into a
single-quoted string by putting a backslash before it:

\begin{lstlisting}
>>> 'Don\'t tread on me'
"Don't tread on me"
\end{lstlisting}

You can add two strings together:

\begin{lstlisting}
>>> 'hello' + ' ' + 'world'
'hello world'
\end{lstlisting}

You can even multiply a string times a number $n \in \Z$ to repeat the
string $n$ times. 

\begin{lstlisting}
>>> 'hello' * 5
'hellohellohellohellohello'
>>> 'hello' * 1
'hello'
\end{lstlisting}

What do you think happens if you multiply the string by $0$? What
about a negative integer? What about a floating-point number? Try it
and see!

\subsection{Booleans}

The other important type (for now) is \term{Booleans}, named after the
English mathematician George Boole. A Boolean is either \code{True} or
\code{False}. They are useful in testing things in programming.

\begin{lstlisting}
>>> 3 > 2
True
>>> 2 > 3
False
\end{lstlisting}

There are six numerical Boolean operators that we'll use:


\begin{center}
  \begin{tabu} to \textwidth {c|c}
    \textbf{Math} & \textbf{Python} \\
    \tabucline \\
    $x < y$ & \code{x < y} \\
    $x \le y$ & \code{x <= y} \\
    $x > y$ & \code{x > y} \\
    $x \ge y$ & \code{x >= y} \\
    $x = y$ & \code{x == y} \\
    $x \ne y$ & \code{x != y} \\
  \end{tabu}
\end{center}
Note, that, in computing, there's an important difference between a
definition $x = y$ (defining $x$ to be equal to $y$), and testing
whether or not $x = y$. In Python, to \emph{define} $x = y$, you write
\code{x = y}. To \emph{test} whether or not $x = y$, you write \code{x == y}.

Also, note that some of the operators apply to things other than
numbers. For instance, we can decide whether or not two strings are
equal:

\begin{lstlisting}
>>> 'hello' == 'hello'
True
>>> 'hello' == 'world'
False
\end{lstlisting}

We have three main operations on Booleans:

\begin{enumerate}
\item \term{Negation}. This is where you just take the opposite of the
  Boolean's value. In mathematical notation, this is $\lnot
  a$. However in Python, it's just \code{not a}.

\begin{lstlisting}
>>> not True
False
>>> not False
True
>>> not ('hello' == 'world')
True
\end{lstlisting}

\item \term{Logical-and}. $a \land b$---or \code{a and b} in
  Python---is true if and only if $a$ and $b$ are both true.

\begin{lstlisting}
>>> True and True
True
>>> True and False
False
>>> False and True
False
>>> False and False
False
\end{lstlisting}

\item \term{Logical-or}. $a \lor b$---or \code{a or b} in Python---is
  true if either $a$ or $b$ is true, or if both are true. This runs
  contrary to our normal semantic definition of ``or'', where either
  $a$ or $b$ can be true, but not both.

\begin{lstlisting}
>>> True or True
True
>>> True or False
True
>>> False or True
True
>>> False or False
False
\end{lstlisting}
\end{enumerate}

Don't worry if you don't completely understand, just keep
moving. You'll get all of this later.  Here are a number of properties
that I'm going to state without proving them. You should convince
yourself that they're true. At the end of the chapter, we'll ``prove''
it, simply by writing a program that does the computation for every
value of $a$ and $b$ (and $c$ for the associative laws).

\begin{enumerate}
\item \code{not (not a) == a}. This is called \term{double negation}.
\item \code{a and (b and c) == (a and b) and c}. This is the
  \term{associative property}
\item \code{a and b == b and a}. This is the \term{commutative
    property}.
\item \code{a and True == a}. This is the \term{conjunctive identity}.
\item \code{a or (b or c) == (a or b) or c}. This is the
  \term{associative property}
\item \code{a or b == b or a}. This is the \term{commutative
    property}.
\item \code{a or False == a}. This is the \term{disjunctive identity}.
\item \code{a and (b or c) == (a and b) or (a and c)}. This is the
  \term{distributive property}.
\item \code{a or (b and c) == (a or b) and (a or c)}. This is another
  \term{distributive property}.
\item \code{not (a and b) == (not a) or (not b)}. This is one of
  \term{De Morgan's laws}.
\item \code{not (a or b) == (not a) and (not b)}. This is the other of
  De Morgan's laws.
\end{enumerate}

\section{Doing things in files}

We're now to the point where we can actually write a program in a
file, and run the program from the command line. I would suggest
making a directory for your Python programs, because we'll be writing
a lot of them.

Open your favorite text editor (you \emph{did} install one earlier,
like I told you, right?), and create a file called \code{hello.py} in
the directory that you created.

Here's the hello world program:

\pyfile{code/hello.py}

The \code{#} is used to indicate a \term{comment}, which is a line
that will be ignored by the Python interpreter. It's used to leave
notes for yourself and other programmers reading your code. You should
get in the habit of describing your program at the top of every
file. It's not required, but it's a good practice. The colors don't
particularly mean anything, they're just there to help with
readability.

\subsection{Running on the command line}

I have positively no idea how to do this in Windows. I've used Unix my
whole life, so I will explain how to do it in Unix. More importantly,
I don't own a machine running Windows. Hopefully, you can figure stuff
out in Windows.

\paragraph{Quick command line tutorial}

As I've mentioned, the dollar sign preceding a line indicates you're
supposed to type the rest of the line in to a terminal. There are four
main commands you need to know:

\begin{enumerate}
\item \code{cd}, which stands for ``change directory.'' It will change
  the current directory in which you are running terminal commands.
\item \code{ls}, which stands for ``list.'' It will list the contents
  of the current directory.
\item \code{pwd}, which stands for ``print working directory.'' It
  will show you the directory you are in.
\item \code{man} pulls up the manual for a program.
\end{enumerate}

Here's a sample, on my machine. Don't type this stuff in, because it
probably won't work on \emph{your} machine. I'm just trying to show an
example of how one might use the command line.

\lstinputlisting{listings/cmdline-tutorial.txt}

You can run \code{man pwd} to view the manual for \code{pwd},
\code{man cd} to view the manual for \code{cd}, \dots, you get the
point. Almost every command has a manual. In the \code{man} interface,
you can use the arrow keys to scroll, and hit \code{q} to exit.

\paragraph{Back to Python}

Now that you know basic command-line navigation, \code{cd} into the
directory where you are storing your code. In my case it's
\code{/home/pharpend/src/pythonnotes/notes/code}. To run the code, run

\lstin{python3hello.txt}

Cool. Let's write a less trivial program. First things first, I need
to explain \term{functions}. In mathematics, a \term{function} is a
construct that takes some input, and produces some output. Critically,
if you give a function the same input, you should get the same
output. That is, \ctext{if $x = y$, then $f(x) = f(y)$.} (That
property is called \term{referential transparency}.)

\textbf{However}: Python bastardizes the term ``function'' to refer to
anything that takes input and produces output, regardless of whether
or not the ``function'' is referentially transparent. For instance,
Python calls this a function:

\lstin{randint-function.txt}

It is clearly \textbf{not} a function, as far as mathematics is
concerned. However, the basic idea is the same. There are some
languages that rely heavily on the use of actual mathematical
functions. Those languages are called \term{functional}
languages. Examples of functional languages are Haskell, Idris, and
OCaml.

In mathematics, a function \ctext{$f : X \to Y$, where $X$ and $Y$ are
  sets,} takes elements from $X$ and maps them to elements in $Y$. In
some programming languages, it's helpful to think of ``functions'' as
maps from elements of one type to elements of another type. The
elements can of course be the same type. Consider this function:

\begin{align*}
  f & : \Z \to \Z \\
  f(x) & = x + 2
\end{align*}

In Python this is:

\pyfile{code/x-plus-2.py}

Python 3 supports ``type annotations'', where you can gently suggest a type
